\documentclass[12pt, letterpaper]{article}
\usepackage[T1,T2A]{fontenc}
\usepackage[russian]{babel}
\usepackage[utf8]{inputenc}
\usepackage{amsmath}
\DeclareMathOperator\erf{erf}
\usepackage{listings}
\usepackage{xcolor, graphicx}
\usepackage{float}
\usepackage{tikz}
\usepackage{hyperref}
\hypersetup{
    colorlinks=true,
    linkcolor=cyan,
    filecolor=magenta,      
    urlcolor=blue,
    pdfpagemode=FullScreen,
    }

\title{Отчёт по лабораторной работе №22 по курсу “Языки и методы программирования”}
\author{Рамалданов Рустамхан Ражудинович}
\begin{document}
\maketitle
\begin{description}
\item\textbf{Студент группы:} \underline{М80-108Б-22 Рамалданов Рустамхан Ражудинович, № по списку 15}    
\item\textbf{Контакты e-mail:} \underline{mrpepelulka@gmail.com}
\item\textbf{Работа выполнена:} \underline{«2» июня 2023 г.}
\item\textbf{Входной контроль знаний с оценкой:} 
\item\textbf{Преподаватель:} \underline{асп. каф. 806 Сахарин Никита Александрович}
\item\textbf{Отчет сдан} \underline{«2» июня 2023 г.}, \textbf{итоговая оценка:}
\item\textbf{Подпись преподавателя:} \underline{\hspace{3cm}}
\end{description}
\newpage
\section{Тема}
Издательская система \TeX{}.
\section{Цель работы}
Получить навыки оформления документов в издательской системе \LaTeX{}.
\section{Задание}
Оформить отчёт об изучении \LaTeX{} на \LaTeX{}.
\section{Оборудование}
\begin{description}
\item\textbf{Процессор:} 11th Gen intel(R) Core(TM) i5-11400H @ 2.70GHz
\item\textbf{ОП:} 16 ГБ
\item\textbf{HDD:} 512 Gb
\item\textbf{Адрес:} 192.168.56.1
\item\textbf{Монитор:} 1920x1080
\item\textbf{Графика:} Nvidia GeForce RTX 3050 Ti Laptop GPU
\end{description}
\section{Программное обеспечение}
\begin{description}
\item\textbf{Операционная система семейства:} linux(ubuntu) версии 5.15.0-47-generic
\item\textbf{Интерпретатор команд:} Visual Studio Code версия 1.76.0
\item\textbf{Текстовый редактор:} Visual Studio Code версия 1.76.0
\end{description}
\section{Идея, метод, алгоритм решения задачи}
Ознакомиться с документацией \LaTeX{} и переписать отчет с Markdown на \LaTeX{}. tex-файл скомпилировать с помощью утилиты pdflatex.
\section{Сценарий выполнения работы}
Продемонстрируем широкий функционал \LaTeX{} на следующих примерах.
\subsection{Пример формул}
\[
\lim\limits_{x \to \infty} \ x^\frac{1}{2} = \infty
\]
\[
& \sqrt[n]{1 + x + x^2 + \cdots + x^n} \sim 1 + \frac{x}{n} &
\]
\[M=
\begin{pmatrix}
a & b & c\\
b & a & c\\
c & b & a
\end{pmatrix}\]
\begin{center}
\begin{tcolorbox}
\begin{center}
\includegraphics[width=0.6\linewidth]{protocol.png}
\end{center}
\end{tcolorbox}
\end{center}
\section{Распечатка протокола}
\begin{lstlisting}[breaklines]
    This is pdfTeX, Version 3.141592653-2.6-1.40.22 (TeX Live 2022/dev/Debian) (preloaded format=latex)
    restricted \write18 enabled.
    entering extended mode
    (./main.tex
    LaTeX2e <2021-11-15> patch level 1
    L3 programming layer <2022-01-21>
    (/usr/share/texlive/texmf-dist/tex/latex/base/article.cls
    Document Class: article 2021/10/04 v1.4n Standard LaTeX document class
    (/usr/share/texlive/texmf-dist/tex/latex/base/size12.clo))
    (/usr/share/texlive/texmf-dist/tex/latex/base/fontenc.sty
\end{lstlisting}  
\section{Дневник отладки}
\begin{tabular}{|c|p{1cm}|p{1.5cm}|c|p{2.5cm}|p{2cm}|p{2.25cm}|}
    \hline
    № & Лаб. или дом. & Дата & Время & Событие & Действие по исправлению & Примечание\\
    \hline
    1 & Дом. & 02.06.23 & 14:00 & Выполнение лабораторной работы & - & -\\
    \hline
\end{tabular}
\section{Замечания автора по существу работы}
\textcolor{red}{Замечания отсутствуют}
\section{Выводы}
Были получены навыки оформления докладов в издательской системе \LaTeX{}. Я убедился в универсальности и удобстве данной системы. Обретенные навыки помогут мне в дальнейшем истользовать \LaTeX{} для написания работ. \\
\flushright \textbf{Подпись студента:} \underline{\hspace{3cm}}
\end{document}